\documentclass{report}

\input{preamble}
\input{macros}
\input{letterfonts}

\setlength{\parindent}{0pt}

\title{\Huge{Physics Notes}\\wahoo}
\author{\huge{Leo Wang}}
\date{}

\begin{document}

\maketitle
\newpage% or \cleardoublepage
% \pdfbookmark[<level>]{<title>}{<dest>}
\pdfbookmark[section]{\contentsname}{toc}
\tableofcontents
\pagebreak

\chapter{1D Kinematics}
\section{Coordinate System in 1D}
Coords in 1D:
\begin{itemize}
  \item Set origin
  \item Set axis
  \begin{itemize}
    \item Set positive coordinate system
  \end{itemize}
  \item Unit vector
  \begin{itemize}
    \item You only need one because it only has one dimension
  \end{itemize}
\end{itemize}

Every point in 1D space has a point $\ihat$ $\lthen$ point $p_1$ has a vector $\ihat_1$ and point $p_2$ has a vector $\ihat_2$

$\ihat_1 = \ihat_2$ $\lthen$ $|\ihat| = 1$

\section{Position Vector in 1D}

$x(t)$ is the position function that changes according to time. 

Position vector is $\vec{r}(t) = x(t)\ihat$

$x(t)$ is the component of the potition vector

$x(t)$ changes sign depending on location $\lthen$ $x(t)>0$ when the object is positive of the origin, $x(t)=0$ when the object is on the vector, and $x(t)<0$ when the object is negative of the origin

Direction of the vector is determined by the sign of the component ($x(t)$) times the component ($\ihat$)

\section{Displacement Vector in 1D}

The position after time $\Delta t$ if the object has been moving at a constant velocity is $r(t + \Delta t) = x(t+\Delta t) \ihat$

Displacement vector for $[t, t + \Delta t]$ is 
\begin{gather*}
\Delta \vec{r} \equiv \vec{r}(t+\Delta t) - \vec{r}(t) = (x(t + \Delta t) - x(t))\ihat\\
\Delta \vec{r} = \Delta x \ihat
\end{gather*}

$\Delta x$ is the component of the displacement vector, where a $\Delta x > 0$ means a displacement in the positive x direction, a $\Delta x = 0$ meaning no net change in position, and a $\Delta x < 0$ meaning a change in position in the negative x direction

\newpage
\section{Avg. Velocity in 1D}
Average velocity depends on the time interval

Ex. for interval $[t,\Delta t]$: 
\begin{gather*}
  \vec{v}_{avg} \equiv \frac{\Delta \vec{r}}{\Delta t} = \frac{\Delta x}{\Delta t} \ihat
\end{gather*}
where $\frac{\Delta x}{\Delta t}$ is the component of velocity 

\section{Instantaneous Velocity in 1D}
How do we find velocity at specific time $t_1$?

Consider avg. velocity over time interval $[t_1,t_{1+\Delta t}]$ $\lthen$ $\vec{v}_{avg} =$ slope of the line.

As we shrink $\Delta t$, we find the slope changes. If we consider $\lim_{x\to 0} \frac{\Delta x}{\Delta t}$, we will get to the slope of the tangent line at time $t_1$

Thus we have:
\begin{gather*}
  \vec{v}(t_1) = \lim_{\Delta t\to 0} \frac{\Delta x}{\Delta t} \ihat \\
  = \lim_{\dlt t \to 0} (\frac{x(t_1 + \Delta t) - x_1 (t)}{\Delta t})\ihat
\end{gather*}

$\vec{v}(t_1)$ is the instantaneous at time $t=t_1$

More generally, $\vec{v}(t) = \lim_{\Delta t \to 0} \frac{\Delta x}{\Delta t} \ihat$ $\lthen$ $\vec{v}(t) = \frac{dx}{dt} \ihat$ 

\newpage
\chapter{1D Kinematics - Acceleration}
\section{Intro to Acceleration}
Acceleration is the change in velocity over time $\lthen$ $\frac{d\vel}{dt}$

For an object in freefall, $\vec{F}_{grav} = m\acc$ where $\acc$ is the gravitational constant

\nt{Freefall is where an object is under the influence of the gravitational force $\vec{F}_g$}

\section{Acceleration in 1D}
$\dlt \vel = (v (t + \dlt t) - v (t))\ihat$ on the interval $[t, \dlt t]$

Now we find instantaneous acceleration:
\begin{gather*}
  \acc (t) = \lim_{\dlt t \to 0} \frac{\dlt \vel}{\dlt t} = \lim_{\dlt t \to 0} \frac{\dlt v}{\dlt t} \ihat
\end{gather*}
Visually, $\acc (t)$ is the slope of the tangent line of the plot of $\vel (t)$ vs $t$

Componently, $\acc (t) = a_x(t) \ihat = \frac{dv_x}{dt} \ihat$

\newpage
\chapter{2D Kinematics}

\section{Coordinate System and Position Vector in 2D}
To represent 2D motion in vectors, one first needs to set up a coordinate system. For any arbitrary point $p_1$, there'll be unit vectors $\ihat_1$ and $\jhat_1$. 
\nt{The Cartesian coordinate system is interesting in that no matter what point we're at, the unit vectors are all the same $\lthen$ we just have $\ihat$ and $\jhat$}

Now we find the position vector, $\vec{r}(t)$, a vector from the origin to the position of the object. We can reprecent $\vec{r}_t$ as two vectors, $x(t)$ and $y(t)$. 

Thus $\vec{r}(t) = x(t)\ihat + y(t)\jhat$

\section{Instantaneous Velocity in 2D}

To find velocity, we want to find $\lim_{\dlt t \to 0} \vec{r}(t)$

Graphically, $\lim_{\dlt t \to 0} \vec{r}(t)$ is direcly tangent to the position at time $t$. 

Given $\vec{r}(t) = x(t)\ihat + y(t)\jhat$, we can find velocity by:
\begin{gather*}
  \vel = \lim_{\dlt t \to 0} \frac{\dlt \vec{r}}{\dlt t} = (\limto \frac{\dlt x}{\dlt t} \ihat) + (\limto \frac{\dlt y}{\dlt t} \jhat) \\
  \vel = \frac{d\vec{r}}{dt} = \frac{dx}{dt} \ihat + \frac{dy}{dt} \jhat \\
  \vel = v_x \ihat + v_y \jhat
\end{gather*}
where $(v_x, v_y) = (\frac{dx}{dt}, \frac{dy}{dt})$

\nt{The speed of the velocity is $v = |(v_x^2 + v_y^2)^{\frac{1}{2}}|$}

We can find the direction of velocity as such:
\begin{gather*}
  \tan \theta = \dfrac{v_y}{v_x}\\
  \theta = \arctan{\dfrac{v_y}{v_x}}
\end{gather*}

\section{Instantaneous Acceleration in 2D}

$\acc (t) - \frac{d\vel}{dt} = \limto \frac{\dlt \vel}{\dlt t}$
\begin{gather*}
  \vel = v_x \ihat + v_y \jhat \lthen \acc = \frac{d\vel}{dt} = \frac{dv_x}{dt} \ihat + \frac{dv_y}{dt} \jhat\\
  \acc = \frac{d^2x}{dt^2} \ihat + \frac{d^2y}{dt^2}\\
  \acc = a_x \ihat + a_y \jhat\\
  a_x = \frac{d^2x}{dt^2} = \frac{dv_x}{dt}, a_y = \frac{d^2y}{dt^2} = \frac{dv_y}{dt}
\end{gather*}
$a = |(a_x^2 + a_y^2)^\frac{1}{2}|$

\section{Projectile Motion}
Projectile motion is the motion of an object while under the influence of gravity. 

Need to apply Newtons 2nd Law in order to analyze the kinematics.
\\
\\
$\jhat$:
\begin{tabular}{|c|c|}
  \hline
  $\vec{F}$ & $m \acc$\\
  \hline
  $-mg$ & $ma_y$\\
  \hline
\end{tabular}
Applying Newtons 2nd Law, we can set $-mg = ma_y$ $\lthen$ $-g = a_y$
\\
\\
$\ihat$: $0 = ma_x$ $\lthen$ $a_x = 0$ 

Thus, we have:

\begin{align}
  &v_y (t) = v_{y0} - gt  \\
  &y(t) = y_0 + v_{y0}t - \half g t^2 \notag\\
  &v_x (t) = v_{x0} \\
  &x(t) = x_0 + v_{x0} t \notag \\
  \notag \\
  &x_0 = 0 \lthen x = v_{x0} t\notag\\
  &t = x/v_{x0}\notag\\
  \notag \\
  &y(x) = y_0 + \frac{v_{y0}x}{v_{x0}} - \half g \frac{x^2}{v_{x0}}
\end{align} 

\newpage

\chapter{Newtons First and Second Laws}
Newtons First Law tells us about the motion of isloated bodies - where the net force is 0. States that an isolated body moves in a straight line at a const velocity. 

Ex: Isloated body at rest will remain at rest as long as its undisturbed. 



\end{document}

